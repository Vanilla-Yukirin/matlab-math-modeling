\documentclass[aspectratio=169]{beamer}
\usepackage{silence}
\WarningFilter{hyperref}{Option `driverfallback' has already been used}
\PassOptionsToPackage{unicode,CJKbookmarks}{hyperref}
\usepackage[UTF8]{ctex}      % 中文支持

% 字体与编码:确保无衬线有真正的 italic 形状,消除 OT1/cmss/... 的warning
\usepackage[T1]{fontenc}            % 西文字体用 T1 编码
\usepackage[default,scale=0.95]{sourcesanspro} % 无衬线换为 Source Sans Pro(含 italic)
% \usepackage[default]{FiraSans}     % 备选:Fira Sans

% 可选:数学直立体用无衬线风格
\usefonttheme{professionalfonts}
% \usepackage{newtxsf}               % 若想数学也更贴近无衬线,可考虑 newtxsf 等;按需启用



% 主题
% https://mpetroff.net/files/beamer-theme-matrix/
\usetheme{Ilmenau}
% 颜色主题 beaver dolphin whale
\usecolortheme{whale}

% 插入图片
\usepackage{graphicx}
\graphicspath{{figs/}}  % 图片直接放到figs目录下

% 调整段间距
\setlength{\parskip}{0.4em}


\newcommand{\CN}[1]{{\CJKfamily{hei}#1}} % 或 song/ Kai 等
% 代码高亮
\usepackage{xcolor}
\usepackage{listings}        % 简单代码渲染
\lstset{
  basicstyle=\ttfamily\small,
  numbers=none,
  breaklines=true,
  showstringspaces=false,
  keywordstyle=\color{blue},
  commentstyle=\color{gray},
  stringstyle=\color{red},
  columns=fullflexible,           % 避免宽度计算问题
  keepspaces=true,                % 保留空格
  texcl=false,                    % 把%后当LaTeX
  escapeinside={(*@}{@*)},        % 代码中插入LaTeX
}

\lstdefinestyle{matCN}{
  basicstyle=\ttfamily\scriptsize, % 小一点的字体
  aboveskip=0.3ex,
  belowskip=-0.7ex,
  language=Matlab,
  texcl=true,         % 只在 Matlab 开
  morecomment=[l]\%,  % Matlab 单行注释
  morestring=[b]",     % 声明双引号字符串,避免其中的 % 被当注释
}

% \usepackage{listingsutf8}    % 支持 utf8
\lstset{inputencoding=utf8}



% 添加代码块标题
\DeclareRobustCommand{\codeinfo}[1]{%
  \nointerlineskip
  {\raggedright\color{gray}\tiny \textbf{文件:}\nolinkurl{#1}}
}

% 信息
\title{Matlab与简单数学模型}
\author{Vanilla\_Yukirin}
\institute{\href{https://github.com/Vanilla-Yukirin/matlab-math-modeling}{Vanilla-Yukirin/matlab-math-modeling}}
\date{\today}

\begin{document}

\begin{frame}
  \titlepage
\end{frame}

\begin{frame}{目录}
  \small
  \begin{columns}[T,totalwidth=\textwidth]
    \begin{column}{0.48\textwidth}
      \tableofcontents[sections={1-3}]
    \end{column}
    \begin{column}{0.48\textwidth}
      \tableofcontents[sections={4-6}]
    \end{column}
  \end{columns}
\end{frame}



\section{Matlab介绍}
\begin{frame}[fragile]{Matlab简介}
  Matlab(矩阵实验室)是由美国MathWorks公司开发的一种高级技术计算语言和交互式环境,广泛应用于科学计算、数据分析、算法开发和可视化等领域。
\end{frame}

\subsection{Matlab与Python的对比}

\begin{frame}[fragile]{选Matlab还是Python}
\begin{columns}[T]
\column{0.5\textwidth}
\textbf{Matlab}
\begin{itemize}
  \item \textbf{内置}数学计算函数和工具箱、交互式工作区、工程领域标准。不需要会环境配置,安装好后开箱即用
  \item 文档齐全,内置帮助文档(按F12即可查看函数详细用法),相比去问AI会更加精确且全面
  \item 付费软件,昂贵的授权费,每个工具箱还需要单独付费\footnote{其实我们学校之前是有订阅过的,但是嘛……},很多学校没有资金订阅
\end{itemize}

\column{0.5\textwidth}
\textbf{Python}
\begin{itemize}
  \item 有numpy、sklearn、matplotlib等科学计算库、jupyter notebook交互式工作区,但是需要掌握环境管理/虚拟环境(anaconda、uv、venv等)
  \item 教程齐全。作为一门\textbf{热门语言},网络上教程很多,问LLM也能得到不错的答案
  \item 免费软件,免费、开源、社区活跃,每个人都能自由的使用(freeware)
\end{itemize}
\end{columns}
\end{frame}

\begin{frame}{为什么建议数模初学者用Matlab}
\begin{itemize}
  \item Matlab 是工程领域的计算器+画图板+方程求解器——5分钟从数据到图表,无需折腾环境配置。安装完立即算矩阵、画图、解方程,让大家专注算法而非工具;
  \item 而 Python 需要先学虚拟环境管理,对完全零基础的同学是额外负担。
  \item 装 Python 的编程环境常见的环境问题(依赖冲突、版本不兼容)往往劝退初学者。
\end{itemize}
\end{frame}

\subsection{Matlab的版本选择与安装}

\begin{frame}{Matlab版本选择}
  Matlab相邻版本之间差别并不大。且近年更新的高级功能做数模一般用不上,所以只要不用过于老的版本(推荐使用 $\ge 2016b$),语法都是通用的。
  
  \textbf{ab尾缀的含义:}
  \begin{itemize}
    \item \textbf{a}:上半年发布的版本,通常在3月份发布
    \item \textbf{b}:下半年发布的版本,通常在9月份发布
  \end{itemize}
  
  通常来说,一个a版本发布后,经过几个月的使用和反馈,MathWorks会修复一些Bug,并在同年秋季的b版本中使其更加稳定。所以\textbf{推荐使用b尾缀的版本}。
  
  接下来的教程是基于Matlab R2024b的,如果你的版本不是2024b也没有关系,操作完全相同。
\end{frame}

\begin{frame}[fragile]{Matlab安装演示}
  自行下载Matlab,网盘链接在钉钉群中
  
  如果下载速度过慢,可以带u盘下课找我拷贝
  
  安装方法:参考公众号安装教程

\end{frame}

\section{Matlab的基础使用}

\subsection{认识Matlab基本界面}

\begin{frame}[fragile]{Matlab基本界面介绍}
  双击Matlab图标,等待一会儿,就会出现白色的Matlab软件窗口。

  \begin{figure}
    \centering
    \includegraphics[width=0.45\textwidth]{Matlab_main.png}
    \caption{Matlab R2024b 界面示例}
  \end{figure}
\end{frame}

\begin{frame}[fragile]{Matlab基本界面介绍}
  \begin{figure}
    \centering
    \includegraphics[width=0.55\textwidth]{Matlab_main_detail.png}
    \caption{Matlab R2024b 界面示例}
  \end{figure}
\end{frame}



\begin{frame}[fragile]{Matlab工作区}
  工作区,顾名思义就是进行工作的地方。建议先更换工作区到准备好的地方,做好文件管理,不要到处都散乱着代码文件。

  通常,初次启动Matlab时,默认路径可能是\texttt{C:\textbackslash Users\textbackslash 用户名\textbackslash Documents\textbackslash MATLAB},建议更改为自己准备好的工作目录。

  \texttt{\small个人习惯在比赛目录下新建一个code或者workspace文件夹专门用来存放代码}
\end{frame}

\begin{frame}[fragile]{更换工作区目录}
  \begin{figure}
    \centering
    \includegraphics[width=0.50\textwidth]{Matlab_openfolder.png}
    \caption{更换当前工作目录}
  \end{figure}
\end{frame}




\begin{frame}[fragile]{命令行窗口}
  命令行窗口的左上角会有三个大于号:\texttt{\lstinline|>>|},表示等待用户输入命令。

  命令行窗口是 Matlab 最常用的窗口,用于输入命令、执行命令、查看结果。

  可以直接在命令行窗口中写一些简单的代码,也可以把它当作一个\textbf{全能计算器}。

  \begin{lstlisting}[style=matCN]
>> format long % set long format
>> pi
ans =
   3.141592653589793
>> pi^2
ans =
   9.869604401089358
>> 
  \end{lstlisting}
\end{frame}

\begin{frame}[fragile]{命令行窗口}
  \begin{figure}
    \centering
    \includegraphics[width=0.4\textwidth]{Command.png}
    \caption{命令行窗口}
  \end{figure}
\end{frame}

\begin{frame}[fragile]{文件命名规范}
  MATLAB 文件名必须以字母开头,最多包含 63 个字母数字字符或下划线。

  也就是说:
  \begin{itemize}
    \item 不能以数字开头
    \item 不能包含中文
    \item 不能用减号 \texttt{-},应使用下划线 \texttt{\_}
  \end{itemize}

养成良好的文件命名习惯,可以帮助我们在数模比赛中快速的找到文件。在数模比赛中,由于会有很多道子问题,所以个人推荐这样命名,既规整又有含义:
  \begin{itemize}
    \item \texttt{problem1\_GA.m} — 问题一,使用遗传算法
    \item \texttt{problem2\_draw.m} — 问题二,画图相关代码
  \end{itemize}
\end{frame}

\begin{frame}[fragile]{运行代码}
  新建 \texttt{HelloWorld.m} 文件,写入如下内容,然后用编辑器的运行按钮或命令行运行。

  \codeinfo{HelloWorld.m}
  \begin{lstlisting}[style=matCN]
clc; % clear console
clear; % clear variables
n=input("Please input a number: ");
n
fprintf("Hello, World!\n")
  \end{lstlisting}

  常见运行方式:
  \begin{enumerate}
    \item 上方的“运行”按钮
    \item 选中代码,按 F9(在命令行中执行所选)
    \item 在行号左侧点击出现的运行蓝条
  \end{enumerate}
\end{frame}

\begin{frame}[fragile]{运行代码示例}
  \begin{minipage}[b]{0.32\textwidth}
    \centering
    \includegraphics[height=0.3\textheight]{Run_code1.png}\\[0.4em]
    \small (a) 运行按钮
  \end{minipage}\hfill
  \begin{minipage}[b]{0.32\textwidth}
    \centering
    \includegraphics[height=0.3\textheight]{Run_code2.png}\\[0.4em]
    \small (b) 选中代码按F9/右键运行
  \end{minipage}\hfill
  \begin{minipage}[b]{0.32\textwidth}
    \centering
    \includegraphics[height=0.3\textheight]{Run_code3.png}\\[0.4em]
    \small (c) 左侧点击运行(运行段)
  \end{minipage}
  \vspace{0.6em}

  运行后命令行会显示“\texttt{Please input a number: }”,
  
  输入一个数字并回车后,会输出你输入的数字并打印 \texttt{Hello, World!}。

\end{frame}

\subsection{Matlab基本语法}

\begin{frame}[fragile]{从矩阵开始}
  Matlab 的名字是“矩阵实验室”,内部数据以矩阵为主,直接面向矩阵进行运算。
  
  \codeinfo{Section2_1_matrix.m}
  \begin{lstlisting}[style=matCN]
clc;clear; % 清空控制台和变量

A=[1,2,3;4,5,6;7,8,9] % 创建矩阵A,是一个3行3列的矩阵
% [] 表示构建一个矩阵
% 逗号或空格用来分隔同一行不同列的元素
% 分号用来分隔不同行(同一列)的元素,或者说表示换行
A(1:2,:) % 拆:取A的前两行
% 矩阵后面连着一个括号,代表取矩阵的某个部分
% 第一个参数 1:2 表示“从1到2”,即第1行和第2行
% 第二个参数 : 表示“所有”
% 所以这里取的是A的前两行,所有列
B=[A;10,11,12] % 添:在A的下面添加一行新数据,构成新矩阵B
% 把新行 [10 11 12] 用分号和A连接,表示把新行添加到A的下面
  \end{lstlisting}
\end{frame}

\begin{frame}[fragile]{向量是特殊的矩阵}
  向量是特殊的矩阵,可以看作是只有一行或者一列的矩阵。

  \texttt{\scriptsize同理,标量可以看作是1行1列的矩阵。这也是为什么说Matlab中万物皆矩阵}

  \codeinfo{Section2_2_vector.m}
  \begin{lstlisting}[style=matCN]
clc;clear
x = [1 2 3 4 5]          % 一行五列 → 行向量
y = [1; 2; 3; 4; 5]      % 五行一列 → 列向量
a = [1,2,3,4,5,6,7,8,9]  % 手写会很累
b = 1:1:10               % 从1开始,步长1,到10结束 → [1 2 3 4 5 6 7 8 9 10]
d = 0:2:20               % 从0开始,步长2,到20结束 → [0 2 4 6 8 10 12 14 16 18 20]
e = 10:-1:1              % 从10开始,步长-1,到1结束 → [10 9 8 7 6 5 4 3 2 1]
  \end{lstlisting}
  步长可正可负,会构造一个等差数列。可以用这种方法快速的构造等间隔的数轴坐标、时间序列等。
\end{frame}


\begin{frame}[fragile]{常用的矩阵创建方法}
  \codeinfo{Section2_3_create.m}
  \begin{lstlisting}[style=matCN]
clc;clear
a=[1,2,3,4,5,6,7,8,9] % 创建一个向量
aa=a'                 % 加'是转置,行→列
b=[1,2,3;4,5,6;7,8,9] % 创建一个矩阵
c=1:1:10              % 创建一个从1到10的向量
e=eye(4)              % 生成4维(4*4)的单位矩阵I(对角线为1)
z=zeros(1,4)          % 生成1行4列的全零矩阵
o=ones(4,1)           % 生成4行1列的全1矩阵
% 随机矩阵
r=rand(4)             % 生成4*4的0-1范围内的随机矩阵
rn=randn(4)           % 生成4*4的均值为0,方差为1的正态分布随机矩阵
ri=randi([1,10],2,4)  % 生成2*4的随机整数矩阵(1到10之间)
% 对角矩阵
d=diag([1,2,3])       % 对角线上是1,2,3的3×3矩阵
d=diag(b)             % 提取b的对角线元素
  \end{lstlisting}
\end{frame}

\begin{frame}[fragile]{常用的矩阵创建方法}
  \codeinfo{Section2_3_create.m}
  \begin{lstlisting}[style=matCN]
% 三角矩阵
U=triu(b)             % 上三角矩阵(下三角变为0)
L=tril(b)             % 下三角矩阵(上三角变为0)
% 生成相同维度的矩阵
size(b)               % 会输出b的尺寸[3 3]
f=zeros(size(b))      % 生成一个和b矩阵尺寸一样的矩阵
% f=zeros(size(b,1),size(b,2)) 和 f=zeros(height(b),width(b)) 同理
% 重复矩阵
R=repmat([1 2],2,3)   % 把[1 2]重复成2行3列块
% 常用序列
x=linspace(0,10,100)  % 0到10之间等距100个点,无需手动设置步长
y=logspace(1,3,5)     % $10^1$到$10^3$之间对数间隔5个点
  \end{lstlisting}
\end{frame}

\begin{frame}[fragile]{变量名和ans} % 变量名和ans
  在刚刚的例子中,我们创建了很多的变量,比如R,x,y等等。

  在Matlab中,变量名可以用字母和数字,但是不能以数字开头。

  如果某一行只是输出结果,但没有给任何变量赋值,那么结果会自动存储在一个名为\texttt{ans}的变量中。

  \codeinfo{Section2_4_variable.m}
  \begin{lstlisting}[style=matCN]
clc;clear;
3+5          % 没有变量名 → 自动存到 ans
ans          % 输出ans查看结果 → 8

score=95     % 有名字 → 存到 score
score        % 查看 → 95
ans          % 此时ans未被修改,还是8
  \end{lstlisting}
\end{frame}


\begin{frame}[fragile]{分号与常用输出方式} % 分号与常用输出方式
  在之前的示例中,可以发现:
  \begin{itemize}
    \item 如果行末不存在分号,那么Matlab会自动输出该行的结果;
    \item 如果行末有分号,则不会输出结果。
  \end{itemize}
  
  \codeinfo{Section2_5_output.m}
  \begin{lstlisting}[style=matCN]
clc;clear;
% 使用分号选择性的输出运算结果
A=[1,2];B=[3,4]; % 不展示A和B
C=A+B            % 展示C(会输出矩阵名)
  \end{lstlisting}
\end{frame}

\begin{frame}[fragile]{分号与常用输出方式}
  使用“无分号”来输出变量,输出的格式较为混乱且不可控(好处是啥都能展示)。在实际的运算中,我们通常会希望仅部分结果,或者按照某种格式输出结果。
  
  下面介绍几种更优雅的输出方式:
  \begin{itemize}
    \item disp() 函数
    \item disp() 与num2str() 函数结合
    \item fprintf() 函数
  \end{itemize}
\end{frame}

\begin{frame}[fragile]{分号与常用输出方式}
  1. disp() 函数

  disp可以说是 MATLAB中的最常用的输出方式。简单直接,适合快速查看变量内容。

  相比于“无分号”直接输出,disp不会输出变量名,只会输出变量的值,更加简洁。

  \codeinfo{Section2_5_output.m}
  \begin{lstlisting}[style=matCN]
% 1. disp
disp(A);         % 打印矩阵(不会输出矩阵名)
disp(pi);        % 打印标量(不会输出变量名)
disp("Ciallo~"); % 打印字符串
% 简单直接,适合快速查看
% 不能混合输出文字和数字
  \end{lstlisting}
\end{frame}

\begin{frame}[fragile]{分号与常用输出方式}
  2. disp() 与 num2str() 函数结合

  disp() 函数也可以和 num2str() 函数结合使用,将数字转为字符串再输出。

  \codeinfo{Section2_5_output.m}
  \begin{lstlisting}[style=matCN]
% 2. num2str
x = 3.14;
disp(['The value of pi is: ' num2str(x)]) % 注意各个字符串之间需要空格/逗号
% 本质上是字符数组的拼接
  \end{lstlisting}

  将会输出:\texttt{The value of pi is: 3.14}
\end{frame}

\begin{frame}[fragile]{分号与常用输出方式}
  3. fprintf() 函数

  fprintf函数是Matlab中功能最强大的输出函数,类似于C语言中的printf函数。

  其整体的结构是:\texttt{fprintf('格式字符串', 变量1, 变量2, ...)}。

  常用的格式符有:\texttt{\%d}、\texttt{\%f}、\texttt{\%s}、\texttt{\%c}、\texttt{\%e}、\texttt{\%g}、\texttt{\%\%}等。

  比如:\texttt{\%.2f}表示输出浮点数并保留两位小数。\texttt{\%d}表示输出整数。

  \codeinfo{Section2_5_output.m}
  \begin{lstlisting}[style=matCN]
% 3. fprintf
x = 3.1415926535;
fprintf('The value of pi rounded to two decimal places is: %.2f\n', x)
fprintf('Today''s temperature is: %d\n', 25) % 需要加\textbackslash n,fprintf不会自动换行
  \end{lstlisting}


\end{frame}


\begin{frame}[fragile]{矩阵的基本运算} % 转置 加减点叉乘除幂
  MATLAB的核心优势就是矩阵运算。掌握这些运算符,你就能像处理数字一样轻松而批量快速的处理矩阵。


  \codeinfo{Sectione2_6_operation.m}
  \begin{lstlisting}[style=matCN]
A = [1 2; 3 4];
A_T = A';    % 加一个单引号 '
  \end{lstlisting}
\end{frame}

\begin{frame}[fragile]{矩阵的基本运算}
  \scriptsize矩阵加减法:对应元素相加减,要求矩阵维度相同。
  

  \codeinfo{Sectione2_6_operation.m}
  \begin{lstlisting}[style=matCN]
A = [1 2; 3 4];
B = [5 6; 7 8];
C = A + B    % [6 8; 10 12]
D = A - B    % [-4 -4; -4 -4]
  \end{lstlisting}

  \scriptsize矩阵乘法和矩阵点乘:矩乘要求矩阵A的列数等于矩阵B的行数;点乘要求矩阵维度相同,对应元素相乘。

  \codeinfo{Sectione2_6_operation.m}
  \begin{lstlisting}[style=matCN]
A = [1 2; 3 4];
B = [5 6; 7 8];
% 矩阵乘法(线代)
C = A * B        % 2x2 * 2x2 = 2x2
% 点乘(对应元素相乘)
D = A .* B       % [1*5 2*6; 3*7 4*8] = [5 12; 21 32]
  \end{lstlisting}


\end{frame}

\begin{frame}[fragile]{矩阵的基本运算}
  \scriptsize矩阵除法和点除

  \codeinfo{Sectione2_6_operation.m}
  \begin{lstlisting}[style=matCN]
A = [2 4; 6 8];
B = [1 2; 3 4];
C = A / B    % 矩阵右除(等价于 A * inv(B))
D = A ./ B   % 点除 [2/1 4/2; 6/3 8/4] = [2 2; 2 2]
% 向左倾斜的是左除,A\textbackslash B等价于inv(A) * B。
  \end{lstlisting}

  \scriptsize矩阵幂和点幂

  \codeinfo{Sectione2_6_operation.m}
  \begin{lstlisting}[style=matCN]
A = [1 2; 3 4];
B = A^2      % A * A (矩阵乘法)
C = A.^2     % [$1^2$ $2^2$; $3^2$ $4^2$] = [1 4; 9 16]
  \end{lstlisting}

  \small总结:矩阵运算符前加点(\texttt{.*}, \texttt{./}, \texttt{.\textasciicircum})表示对应元素操作。否则则为矩阵操作。如果出现运算的报错,优先检查是不是忘记加点了

\end{frame}

\begin{frame}[fragile]{矩阵的基本运算}
  \scriptsize 矩阵元素的访问与修改

  \codeinfo{Sectione2_6_operation.m}
  \begin{lstlisting}[style=matCN]
% matrix(row,col) :访问row行,col列的元素,修改可直接加上赋值号。
% matrix(num) :按照顺序访问,特别注意,matlab一列一列的访问,与其它语言不同。
% matrix(row,:) :访问第row行的所有元素。
% matrix(row,:)=[1,2,……] :修改第row行,直接赋值覆盖原值。
% matrix(row,:)=[] :删除第row行。
% matrix(:,col) :表示第col列的相关操作,和行一致。
matrix=[1,2,3,4;5,6,7,8]
[matrix(1) matrix(2) matrix(3)] % 发现单参数访问顺序是 1 5 2 6 3 7 5 8
matrix(2,4) % 第2行第4列
matrix(1,:) % 第一行,所有列,即第一行一整行
matrix(:,1) % 所有行,第一列,即第一列一整列
matrix(1,1)=100 % 修改第一行第一列为100
temp=matrix; temp(1,:)=[] % 删除第一行。同时矩阵会从2*4变成1*4
temp=matrix; temp(:,2)=[] % 删除第二列。同时矩阵会从2*4变成2*3
  \end{lstlisting}
\end{frame}


\begin{frame}[fragile]{Matlab中的循环与条件语句}
  \scriptsize作为一门编程语言,为什么这么晚才讲循环与条件?因为Matlab的核心思想就是“向量化>循环”,能用矩阵运算的就不要用循环。

  \scriptsize下面简单介绍循环和条件语句的写法。

  \codeinfo{Section2_7_loop_condition.m}
  \begin{lstlisting}[style=matCN]
score = 85;
if score >= 90
    disp("(*@\CN{优秀}@*)")
elseif score >= 60
    disp("(*@\CN{及格}@*)")
else
    disp("(*@\CN{挂科}@*)")
end
  \end{lstlisting}

  \scriptsize if后直接跟着条件,以end结束。elseif和else可选。

  \scriptsize用\texttt{==}表示等于而不是赋值\texttt{=}。此外,有\texttt{\&}表示逻辑与,\texttt{|}表示逻辑或,\texttt{\textasciitilde}表示逻辑非。
\end{frame}

\begin{frame}[fragile]{Matlab中的循环与条件语句}

  \codeinfo{Section2_7_loop_condition.m}
  \begin{lstlisting}[style=matCN]
sum_val = 0;
for i = 1:100
    sum_val = sum_val + i;  % 计算1+2+...+100
end
disp(sum_val)  % 结果:5050

% 错误示范:在循环中动态扩展数组
A = []; % 应该写成 A=zeros(10000,1);
for i = 1:10000
    A(i) = i;
end
% 会给警告:变量的大小似乎在(脚本内的)每个循环迭代都会更改。请考虑进行预分配以提升速度。
  \end{lstlisting}

  \scriptsize特别地,在for中尽量避免动态地扩展数组。这是因为循环中每次迭代都要重新分配内存,这会耗费大量时间。

\end{frame}


\begin{frame}[fragile]{Matlab中的循环与条件语句}
  推荐用\textbf{向量化}代替循环。

  
  可以直接用sum函数计算1到100的和,而不需要循环。

  对于简单的条件语句,也可以用逻辑缩索引来代替if。

  \codeinfo{Section2_7_loop_condition.m}
  \begin{lstlisting}[style=matCN]
% 判断成绩等级
scores = [85, 92, 45, 78];  % 4个学生的成绩
pass = scores >= 60;        % [1, 1, 0, 1] 逻辑向量
disp(scores(pass))          % 直接输出及格分数:[85,92,78]

% 计算1+2+...+100
sum_val = sum(1:100);  % 1行代码 vs 5行循环
  \end{lstlisting}
\end{frame}




\begin{frame}[fragile]{Matlab中的函数定义与调用}
  Matlab 支持很多形式:匿名函数(\texttt{f=@(x)x.\textasciicircum2})、文件内多函数、嵌套函数、类方法……
  但入门阶段只需要掌握\textbf{最常用的一种:单文件函数}。


  \begin{figure}
    \centering
    \includegraphics[height=0.5\textheight]{Akari.jpeg}
  \end{figure}
\end{frame}

\begin{frame}[fragile]{Matlab中的函数定义与调用}
  \codeinfo{mySum.m}
  \begin{lstlisting}[style=matCN]
function s = mySum(a, b) % 直接以function开头,且函数名与文件名完全一致
% 传入了两个参数a和b
s = a + b; % s是函数的返回值,在遇到function的end时会将s作为函数的值返回
end
  \end{lstlisting}
  
  \scriptsize 上面定义了一个简单的函数mySum,功能是计算两个数的和。

  \codeinfo{Section2_8_function.m}
  \begin{lstlisting}[style=matCN]
% 需要保证与函数文件mySum.m在同一个目录下
result = mySum(10, 20);
  \end{lstlisting}

  \small 函数的三个要点:
  \begin{itemize}
    \item \scriptsize \texttt{function 输出 = 函数名(输入)}
    \item \scriptsize 文件名必须和函数名相同
    \item \scriptsize 用\texttt{end}结束
  \end{itemize}

\end{frame}


\begin{frame}[fragile]{Matlab中的函数定义与调用}
  \scriptsize 函数内部可以写多行代码,也可以返回多个变量。

  \codeinfo{stat2.m}
  \begin{lstlisting}[style=matCN]
function [m, s] = stat2(x)
m = mean(x); % 计算均值
s = std(x);  % 计算标准差
end
  \end{lstlisting}
  
  \scriptsize 调用:

  \codeinfo{Section2_9_function.m}
  \begin{lstlisting}[style=matCN]
function [m, s] = stat2(x)
[data_mean, data_std] = stat2([1,2,3,4,5])
% 函数也支持向量化运算,可以实现输入向量或矩阵得到批量运算结果。
[datas_mean, datas_std] = stat2([1,2,3;4,5,6;7,8,9])
end
  \end{lstlisting}

  \small 函数适合:重复计算、逻辑封装、让脚本更清晰。

\end{frame}

\begin{frame}[fragile]{符号运算}
  \scriptsize 符号运算:让 Matlab 像在纸上做\textbf{代数、微积分推导},而不是只是计算小数结果。

  \codeinfo{Section2_10_symb.m}
  \begin{lstlisting}[style=matCN]
clc;clear
% 1. 定义一个符号变量和函数 $f(x)$
syms x              % 告诉 Matlab:$x$ 是“符号”
f = x^3 - 2*x + 1;  % 像在高数里写函数
% 2. 求导 $f'(x)$
df = diff(f)        % 结果:$3\times x^2 - 2$
% 3. 求不定积分 $\int f(x) dx$
F = int(f)          % 结果:$x^4/4 - x^2 + x$
% 4. 求定积分 $\int_0^1 f(x) dx$
I = int(f, x, 0, 1) % 得到一个精确结果(分数)
% 5. 在 $x = 2$ 处代入
val = subs(f, x, 2) % 把 x 换成 2 代入
  \end{lstlisting}

  \scriptsize 这块和大家的\textbf{高等数学 / 数学分析}联系很紧:以后遇到复杂积分、推公式,可以先让 Matlab 帮你验算。
\end{frame}


\subsection{Matlab基础绘图}
\begin{frame}[fragile]{绘图的作用}
  \scriptsize 绘图是数模与科研中必不可少的一环,可以帮助我们:
  \begin{itemize}\scriptsize
    \item 快速观察数据趋势(如上升、震荡、周期性);
    \item 验证模型是否合理;
    \item 制作论文与报告中的结果图像。
  \end{itemize}

  \scriptsize Matlab 中最简单的绘图方式:只给 \texttt{y},默认横轴为 1,2,3,...

  \codeinfo{Section2_11_plot.m}
  \begin{lstlisting}[style=matCN]
clc;clear;close all; % close用来关闭之前生成的绘图窗口
y = [2, 4, 3, 5, 6, 4];
plot(y);               % 本质上是默认 x = 1:length(y)
  \end{lstlisting}

  只需要简单的一个\texttt{plot(y)},就能画出数据的折线图。

  % 绘图效果展示
  \begin{figure}
  \centering
  % \includegraphics[width=0.8\textwidth]{Plot_example.png}
  \end{figure}
\end{frame}

\begin{frame}[fragile]{plot(x, y):绘制自变量与因变量}
  更常用的绘图方式:指定 x 和 y。

  \codeinfo{Section2_11_plot.m}
  \begin{lstlisting}[style=matCN]
clc;clear;close all;
x = 0:0.1:2*pi;      % 自变量
y = sin(x);          % 因变量
plot(x, y);          % 绘制 y = sin(x)
  \end{lstlisting}

  要点:
  \begin{itemize}
    \item x 与 y 必须长度一致;
    \item x 可以是任意数值(时间、空间、数据点);
    \item 适合用来绘制函数曲线或实验数据。
  \end{itemize}

\end{frame}


\begin{frame}[fragile]{图像标注:标题与坐标轴}
  一张好的图要\textbf{能说明问题,展示数据的含义}。因此在图中,做好各处标注很重要。

  \codeinfo{Section2_11_plot.m}
  \begin{lstlisting}[style=matCN]
clc;clear;close all;
x = 0:0.1:2*pi;
y = sin(x).^2;
plot(x, y)
title("y = sin(x)^2")    % 标题
xlabel("x (rad)")      % x轴标签
ylabel("y")            % y轴标签
legend("sin(x)^2")       % 图例(说明线条含义)
  \end{lstlisting}

  特别提醒一下,画图模块的字符串大多都是支持LaTeX的,比如可以用\texttt{\textbackslash sin(x)\textasciicircum 2}来表示数学公式。但是有时候也会触发奇怪的错误,导致标签显示不正常。

\end{frame}

\begin{frame}[fragile]{图像窗口控制:figure}
  \scriptsize Matlab 默认在同一个窗口中绘图,如果想创建新的图像窗口,需要使用 \texttt{figure}。


  \codeinfo{Section2_11_plot.m}
  \begin{lstlisting}[style=matCN]
clc;clear;close all;
x = 0:0.1:2*pi;

figure;          % 新建一个绘图窗口
plot(x, sin(x))
title("(*@\CN{这是第一张图}@*)")

figure;          % 再打开一个新窗口
plot(x, cos(x))
title("(*@\CN{这是第二张图}@*)")
  \end{lstlisting}


\end{frame}

\begin{frame}[fragile]{在同一张图中绘制多条曲线}
  \scriptsize 默认情况下,新的 plot 会覆盖旧图。可以使用 \texttt{hold on} 可以在同一张图中添加多条线。
  
  \codeinfo{Section2_11_plot.m}
  \begin{lstlisting}[style=matCN]
clc;clear;close all;
x = 0:0.1:2*pi;

hold on          % 开始叠加
plot(x, sin(x))
plot(x, cos(x))
hold off         % 结束叠加

legend("sin(x)", "cos(x)")
  \end{lstlisting}

  \scriptsize 在数模中常用于:
  \begin{itemize}\scriptsize
    \item 比较真实数据 vs 模型预测数据
    \item 比较不同参数的模型结果
    \item 展示多个方案的曲线
  \end{itemize}
\end{frame}

\begin{frame}[fragile]{线型、颜色与标记控制}
  美感也是论文追求的目标之一。Matlab 支持丰富的绘图样式,可用于论文美化。

  \codeinfo{Section2_11_plot.m}
  \begin{lstlisting}[style=matCN]
clc;clear;close all;
x = 0:0.1:2*pi;

plot(x, sin(x), 'r--', 'LineWidth', 1.5)   % 红色、虚线、加粗
hold on
plot(x, cos(x), 'b-o', 'MarkerSize', 4)    % 蓝色、圆点标记
hold off

legend("sin(x)", "cos(x)")
  \end{lstlisting}

  \scriptsize plot函数有很多的样式控制方法,具体可以右键plot,打开关于plot函数的帮助,查看官方文档。常见样式参数:
  \begin{itemize}\scriptsize
    \item 颜色:'r'红, 'b'蓝, 'g'绿, 'k'黑…
    \item 线型:'-'实线, '--'虚线, ':'点线
    \item 标记:'o'圆点, '*'星, 's'方块
  \end{itemize}

  \end{frame}

% \begin{frame}[fragile]{}

%   \begin{lstlisting}[style=matCN]


%   \end{lstlisting}
% \end{frame}


\section{Matlab简单应用} % 给两道题目

\subsection{矩阵运算} % 简单的题目

\begin{frame}[fragile]{问题1}
  \scriptsize 1. 请用分别一行代码,生成以下矩阵:
  \begin{itemize}
    \item 3行4列的随机矩阵
    \item 3行10列的矩阵,元素为1到30的连续整数
    \item 下三角部分为[0,1]上是均匀随机数,其余为全1的5行5列矩阵
  \end{itemize}

  (提示:请使用help自学reshape函数)

  2. 已知\texttt{x=[1 2 3]; y=[4 5 6];},请用一行代码计算两个向量的点积(提示:元素相乘再求和)

  3. 已知\texttt{A=[3, -1, 4, -2, 5];},请用一行代码把A中负数变成0,正数不变(提示:用逻辑索引)
\end{frame}

\begin{frame}[fragile]{问题1答案}
  \codeinfo{Section3_1_ans.m}
  \begin{lstlisting}[style=matCN]
% 3行4列的随机矩阵
A = rand(3, 4)

% 3行10列的矩阵,元素为1到30的连续整数
B = reshape(1:30, 3, 10)

% 下三角部分为[0,1]上是均匀随机数,其余为全1的5行5列矩阵
C = ones(5) + tril(rand(5)-1)

% 计算两个向量的点积
x = [1 2 3]; y = [4 5 6];
sum(x .* y)

% 负数变0,正数不变
A = [3, -1, 4, -2, 5];
A(A<0) = 0
  \end{lstlisting}
\end{frame}




\subsection{递推计算} % 斐波那契

\begin{frame}[fragile]{问题2}
  \textbf{计算斐波那契数列第N项的最低4位数,即:}

  $$F_0=0, F_1=1, F_n = F_{n-1} + F_{n-2}$$
  
  求 $F_N \mod p$

  其中,$N = 40$,$p = 10^4$

  \textbf{Tips: 可以尝试用“递推+for循环”和“递归+函数”两种写法实现}

  \texttt{\scriptsize如果$N=10^8$呢?如果$N=10^{18}$呢?同时看看,你的程序需要多长时间计算出结果?}
  
  \scriptsize 推荐阅读:\href{https://oi-wiki.org/math/combinatorics/fibonacci/}{oi-wiki: 斐波那契数列},\href{https://zhuanlan.zhihu.com/p/646539368}{zhihu: Pisano 周期}
\end{frame}

\begin{frame}[fragile]{问题2答案}

  \codeinfo{Section3_2_fib.m}
  \begin{lstlisting}[style=matCN]
clc;clear;
% 计算斐波那契数列第N项模p的值
N=40;
p=10000;
% 递推关系:F(i) = F(i-1) + F(i-2)
tic
A1 = 1; % F(1),即前一项 -> 在迭代过程中代表F(i-1)
A2 = 1; % F(2),即当前项 -> 在迭代过程中代表F(i)
for i=3:N % 然后这里只需要从第3项开始迭代,每次迭代生成第i项
    t=A2; % 临时存储一下当前项A2,其在下一轮会变成前一项A1
    A2=mod(A1+A2,p); % 递推出新的一项,并取模防止溢出
    A1=t; % 将原来的当前项作为下一轮的前一项
end
result=A2; % F(N)
disp(result);
toc
  \end{lstlisting}
\end{frame}


\subsection{数值计算} % 方程求解+积分+微分

\begin{frame}[fragile]{问题3}
  \scriptsize下面给出一个函数:

  $$ f(x) = x^2 e^{-x} $$
  
  请完成以下任务(每题都要求用\textbf{一行代码}完成):

  \begin{itemize}
    \item (1) 求 $f(x)$ 的导数 $f'(x)$
    \item (2) 计算定积分 $\int_0^3 f(x)\,dx$
    \item (3) 在 $x = 2$ 处,求 $f(2)$ 的数值
    \item (4) 解方程 $f(x)=0.1$,求正根(提示:使用 \texttt{vpasolve} 或 \texttt{fzero})
    \item (5) 画图验证(4)的正确性
  \end{itemize}

\end{frame}

\begin{frame}[fragile]{问题3答案}

  \codeinfo{Section3_3_symb.m}
  \begin{lstlisting}[style=matCN]
clc;clear;
% 定义函数
syms x
f = x.^2 * exp(-x)
%求 $f(x)$ 的导数 $f'(x)$
df = diff(f)
% 计算定积分 $\int_0^3 f(x)\,dx$
int(f, x, 0, 3)
% 在 $x = 2$ 处,求 $f(2)$ 的数值
subs(f, x, 2)
  \end{lstlisting}
\end{frame}

\begin{frame}[fragile]{问题3答案}

  \codeinfo{Section3_3_symb.m}
  \begin{lstlisting}[style=matCN]
% 解方程 $f(x)=0.1$,求正根
format long
r1 = vpasolve(f == 0.1, x, 1)   % 从 1 附近找一个解
r2 = fzero(@g, 1)               % 注意:g 里已经减掉 0.1
% 画图验证
figure
xx = 0:0.01:5;
yy = xx.^2 .* exp(-xx);
hold on
plot(xx, yy)                     % 曲线 y = f(x)
plot(xx, 0.1*ones(size(xx)))     % 水平线 y = 0.1
plot(double(r1), double(subs(f,x,r1)), 'ro')  % r1 对应的交点
hold off
function y = g(x) % 函数也可以定义在文件的末尾
    y = x.^2 .* exp(-x) - 0.1;
end
  \end{lstlisting}
\end{frame}






\section{简单数学模型}

\subsection{线性回归}

\begin{frame}[fragile]{引例}
  \small 我们得到了一组数据。对于这16个样本,我们好奇:身高和腿长的关系是什么?

{
\scriptsize
\centering
\begin{tabular}{c|*{16}{c}}
\textbf{身高} & 143 & 145 & 146 & 147 & 149 & 150 & 153 & 154 & 155 & 156 & 157 & 158 & 159 & 160 & 162 & 164 \\
\hline
\textbf{腿长} &  88 &  85 &  88 &  91 &  92 &  93 &  93 &  95 &  96 &  98 &  97 &  96 &  98 &  99 & 100 & 102 \\
\end{tabular}
}

  \small 于是做出了散点图,发现“看上去像是一条线,但不完全是直线”

  \small 能不能让电脑帮我们找一条最接近所有点的直线呢?

  \codeinfo{Section4_1_example.m}
  \begin{lstlisting}[style=matCN]
clc; clear;
x = [143 145 146 147 149 150 153 154 155 156 157 158 159 160 162 164];
y = [ 88  85  88  91  92  93  93  95  96  98  97  96  98  99 100 102];
figure;
plot(x, y, 'bo', 'MarkerSize',6, 'LineWidth',1.5);
grid on; % 打开背景的网格,看的更清楚
  \end{lstlisting}
\end{frame}

\begin{frame}[fragile]{最接近所有数据点的直线}
  \begin{figure}
  \centering
  \includegraphics[height=0.8\textheight]{S4_1_example.png}
  \end{figure}
\end{frame}


\begin{frame}[fragile]{一元线性回归模型}
  \small 观察散点图后,我们希望用一条“直线”去近似这种关系:

  $$y = \beta_0 + \beta_1 x + \varepsilon$$

  其中:
  \begin{itemize}\scriptsize
    \item $\beta_0$:截距(当身高为0时腿长的理论值,在本例中主要是数学意义);
    \item $\beta_1$:斜率,表示身高每增加1cm,腿长平均增加多少;
    \item $\varepsilon$:误差项,表示“个体差异”“测量误差”等噪声。
  \end{itemize}

  \normalsize 一元线性回归的目标:\textbf{找到一组$\hat\beta_0,\hat\beta_1$,使得这条直线“尽可能贴近所有点”。}

\end{frame}


\begin{frame}[fragile]{最小二乘}

  \small 如何衡量“一条直线是否接近所有点”?

  \small 对于第 $i$ 个样本,有真实值 $y_i$,预测值 $\hat y_i = \hat\beta_0 + \hat\beta_1 x_i$,它们之间的“差”称为\textbf{残差}:
  $$
    e_i = y_i - \hat y_i
  $$

  \small 最小二乘法(Least Squares)的思想:
  $$
    \text{选}~\hat\beta_0,\hat\beta_1 \text{使得}
    \quad \sum_{i=1}^{n} e_i^2
    = \sum_{i=1}^{n} (y_i - \hat\beta_0 - \hat\beta_1 x_i)^2
    \quad \text{最小}.
  $$
  
  \small 不做详细推导,只需要知道:\textbf{最小二乘法给出了一条“总体误差平方和最小”的直线},这就是 Matlab 中 \texttt{regress} 命令所做的事。
\end{frame}


\begin{frame}[fragile]{regress命令}
  \small 我们可以把一元线性回归写成矩阵形式:
  $$
    Y = X \beta,\quad
    X = \begin{bmatrix}
      1 & x_1 \\
      1 & x_2 \\
      \vdots & \vdots \\
      1 & x_n
    \end{bmatrix},\quad
    \beta = \begin{bmatrix}\beta_0 \\ \beta_1\end{bmatrix}.
  $$

  \small 在 Matlab 中,只需几行代码就可以完成回归分析:

  \codeinfo{Section4_2_regress.m}
  \begin{lstlisting}[style=matCN]
[b, bint, r, rint, stats] = regress(y, X);
  \end{lstlisting}

  \scriptsize 其中 \texttt{b} 给出 $\hat\beta_0,\hat\beta_1$ 的估计;\texttt{stats} 中包含拟合优度 $R^2$、F 值以及 p 值。
\end{frame}

\begin{frame}[fragile]{regress命令}
  \small 在 Matlab 中,只需几行代码就可以完成回归分析:

  \codeinfo{Section4_2_regress.m}
  \begin{lstlisting}[style=matCN]
% 显示结果
fprintf("(*@\CN{截距}@*)beta0 = %.2f\n", b(1));
fprintf("(*@\CN{自变量}@*)beta1 = %.2f\n", b(2));
fprintf("(*@\CN{残差向量}@*) r\n"); disp(r);
fprintf("(*@\CN{残差区间}@*) rint\n"); disp(rint); % 若区间跨过0,则该点可能为异常点。
% 回归总体统计量 stats
fprintf("R^2(*@\CN{(拟合优度)}@*)= %.4f\n", stats(1));
fprintf("F (*@\CN{统计量}@*) = %.4f\n", stats(2));
fprintf("p (*@\CN{值}@*) = %.4e\n", stats(3));
  \end{lstlisting}

\end{frame}


\begin{frame}[fragile]{回归结果的解释}
  \small 运行上一页的代码,得到的结果为:
  $$
    \hat\beta_0 \approx -16.07,\quad
    \hat\beta_1 \approx 0.72,\quad
    R^2 \approx 0.9282.
  $$

  \small 可以这样理解:
  \begin{itemize}\scriptsize
    \item \textbf{斜率 $\hat\beta_1 \approx 0.72$}:身高每增加 1cm,腿长平均增加约 0.7cm;
    \item \textbf{截距 $\hat\beta_0 \approx -16$}:在本例中主要是数学量,不必过度解读其现实意义;
    \item \textbf{$R^2 \approx 0.9282$}:说明直线解释了约 92.82\% 的 $y$ 变化,拟合效果非常好。
  \end{itemize}

  \small 此外,\texttt{bint} 给出回归系数的置信区间。\texttt{stats(3)} 中的 p 值若小于 0.05,说明“身高对腿长的线性影响”在统计上是显著的。
\end{frame}


\begin{frame}[fragile]{预测与拟合直线的绘制}
  \small 得到回归系数后,就可以计算每个样本点的预测值 $\hat y = \hat\beta_0 + \hat\beta_1 x$,并画出拟合直线:
  
  \codeinfo{Section4_3_plot.m}
  \begin{lstlisting}[style=matCN]
clc; clear;
x = [143 145 146 147 149 150 153 154 155 156 157 158 159 160 162 164]'; % 注意需要先转置,变成列向量
y = [ 88  85  88  91  92  93  93  95  96  98  97  96  98  99 100 102]';
X = [ones(length(x),1), x];
b = regress(y, X);
y_hat = X * b;  % 预测值
figure;
plot(x, y, 'k+', 'MarkerSize',6); hold on;  % 原始数据
plot(x, y_hat, 'r-', 'LineWidth',1.5);      % 回归直线
legend("(*@\CN{原始数据}@*)","(*@\CN{回归直线}@*)","Location","best");
grid on;
  \end{lstlisting}

\end{frame}


\begin{frame}[fragile]{预测与拟合直线的绘制}
  \begin{figure}
  \centering
  \includegraphics[height=0.6\textheight]{S4_3_plot.png}
  \end{figure}

  \small 在建模与论文写作中,这种“散点 + 拟合直线”的图是非常常见的展示方式。
\end{frame}


\begin{frame}[fragile]{多元线性回归}
  \small 刚才我们只考虑了一个自变量(身高)。如果有多个影响因素,例如:
  \begin{itemize}\scriptsize
    \item $x_1$:身高,$x_2$:体重,$x_3$:年龄
  \end{itemize}
  那么就得到\textbf{多元线性回归模型}:
  $$ y = \beta_0 + \beta_1 x_1 + \beta_2 x_2 + \beta_3 x_3 + \varepsilon. $$
  
  \small 在 Matlab 中,只需要把 $X$ 矩阵 从两列扩展为多列,其余用法完全相同——调用 \texttt{regress} 命令即可得到结果。
  
  \codeinfo{Section4_4_multi.m}
  \begin{lstlisting}[style=matCN]
clc; clear;
% 数据在这里忽略
X = [ones(length(x1),1), x1, x2, x3];
[b, ~, ~, ~, stats] = regress(y, X)
  \end{lstlisting}
\end{frame}


\begin{frame}[fragile]{线性回归小结}
  本节学习了:
  \begin{itemize}\small
    \item 一元线性回归模型的基本形式与含义
    \item 最小二乘法思想
    \item 如何使用 \texttt{regress} 命令完成回归分析
    \item 解释回归结果并画出拟合直线
    \item 拓展多元线性回归
  \end{itemize}

  \tiny推荐自学:
  \begin{itemize}
    \item 线性回归求解本质:用矩阵除法也可以解决线性回归问题
    \item 评估回归效果:$R^2$的含义、F检验、t检验、p值、置信区间、残差分析
    \item 预测方式:点预测,区间预测
    \item \texttt{stepwise} 逐步回归,\texttt{polyfit polyval} 多项式回归,\texttt{rstools} 多元二项式回归,\texttt{nlinfit} 非线性回归
  \end{itemize}
  
\end{frame}

\subsection{非线性规划模型与启发式算法求解}


\begin{frame}[fragile]{}

  \codeinfo{}
  \begin{lstlisting}[style=matCN]


  \end{lstlisting}
\end{frame}

\begin{frame}[fragile]{}

  \codeinfo{}
  \begin{lstlisting}[style=matCN]


  \end{lstlisting}
\end{frame}


\section{课后习题}

\subsection{分析某网站用户增长情况}

\begin{frame}[fragile]{问题背景}
  \scriptsize 你连续爬取了某网站的用户数量信息,记录了:

  \texttt{year, month, day, hour, minute, second, number}。

\scriptsize 数据时间跨度:2025-11-04 ~ 2025-11-13,共约 427 条记录。

  \codeinfo{homework1/data.csv}
  \begin{lstlisting}[style=matCN]
2025,11,4,4,59,22,3825
2025,11,4,5,20,47,3826
2025,11,4,5,40,48,3827
...
  \end{lstlisting}

\end{frame}

\begin{frame}[fragile]{问题1-1,1-2}
  \small \textbf{任务 1:数据读取与预处理}
  \begin{itemize}\scriptsize
    \item 使用 \texttt{readmatrix} 读取 CSV 文件,并保存为矩阵
    \item 查看数据结构(行数、列数)
  \end{itemize}

  \vspace{4pt}
  \small \textbf{任务 2:基于时间秒数的一元线性回归}
  \begin{itemize}\scriptsize
    \item 将时间字段转换为“从 2025-11-04 00:00:00 起经过的总秒数”\texttt{t}
    \item 构造模型:$number = \beta_0 + \beta_1 \cdot t + \varepsilon$
    \item 使用 \texttt{regress()} 求参数与统计量
    \item 绘制“散点图 + 回归直线”;
    \item 根据模型预测:网站何时用户数达到 \textbf{10000}?
  \end{itemize}
\end{frame}

\begin{frame}[fragile]{问题1-2结果展示}

  \begin{figure}
    \centering
    \includegraphics[height=0.75\textheight]{linear_prediction.png}
  \end{figure}
\end{frame}

\begin{frame}[fragile]{问题1-3:附加题}
  \small 观察数据发现:网站用户量的日内增长速度存在周期性。

  \small \textbf{任务 3:引入多元变量建模}
  \begin{itemize}\scriptsize
    \item 构造两个自变量:
      \begin{itemize}\scriptsize
        \item \texttt{day\_index}:从起始日开始的第几天(0、1、2...)
        \item \texttt{minute\_of\_day}:一天中的第几分钟(hour*60 + minute)
      \end{itemize}

    \item 构造模型:
    $$number = \beta_0 + \beta_1 day + \beta_2 minute + \varepsilon$$

    \item 若拟合效果不足,可尝试加入非线性项
    \item 使用改进模型预测:网站何时用户数将达到 \textbf{10000}?
    \end{itemize}
\end{frame}

\begin{frame}[fragile]{问题1-4:开放题}
  \small \textbf{任务 4:}
  
  \small 基于你的分析与模型结果,请讨论:

  \begin{itemize}\scriptsize
    \item 你认为这个网站的主要功能可能是什么?(依据增长节奏)
    \item 从数据趋势推测用户画像(年龄段、使用习惯等);
    \item 数据中的周期性说明了怎样的用户行为特征?
    \item 如果你是该站站长的朋友,你会提出哪些运营建议?
  \end{itemize}

  \small 本题无标准答案,重在通过建模结果进行合理推断。
\end{frame}


\subsection{非线性规划}

\begin{frame}[fragile]{}

  \codeinfo{}
  \begin{lstlisting}[style=matCN]


  \end{lstlisting}
\end{frame}

\section{总结与答疑}


\begin{frame}[fragile]{总结}
\small
  \begin{columns}[T,totalwidth=\textwidth]
    \begin{column}{0.48\textwidth}
      \tableofcontents[sections={1-3}]
    \end{column}
    \begin{column}{0.48\textwidth}
      \tableofcontents[sections={4-6}]
    \end{column}
  \end{columns}
\end{frame}

\begin{frame}[fragile]{总结}
今天晚上和周六周日,我会在数模5群里面看大家的问题

大家可以把不懂的地方发到群里,我和数模协会的同学会尽量帮大家解答

\end{frame}

\end{document}