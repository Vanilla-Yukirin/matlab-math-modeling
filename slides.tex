\documentclass[aspectratio=169]{beamer}
\usepackage{silence}
\WarningFilter{hyperref}{Option `driverfallback' has already been used}
\PassOptionsToPackage{unicode,CJKbookmarks}{hyperref}
\usepackage[UTF8]{ctex}      % 中文支持

% 字体与编码:确保无衬线有真正的 italic 形状,消除 OT1/cmss/... 告警 ——
\usepackage[T1]{fontenc}            % 西文字体用 T1 编码
\usepackage[default,scale=0.95]{sourcesanspro} % 无衬线换为 Source Sans Pro(含 italic)
% \usepackage[default]{FiraSans}     % 备选:Fira Sans

% 可选:数学直立体用无衬线风格
\usefonttheme{professionalfonts}
% \usepackage{newtxsf}               % 若想数学也更贴近无衬线,可考虑 newtxsf 等;按需启用



% 主题
% https://mpetroff.net/files/beamer-theme-matrix/
\usetheme{Ilmenau}       % 主题
\usecolortheme{whale}        % 颜色主题 beaver dolphin whale

% 代码高亮
\usepackage{xcolor}
\usepackage{listings}        % 简单代码渲染
\lstset{
  basicstyle=\ttfamily\small,
  numbers=none,
  breaklines=true,
  showstringspaces=false,
  keywordstyle=\color{blue},
  commentstyle=\color{gray},
  stringstyle=\color{red}
}

% 信息
\title{Matlab与简单数学模型}
\author{何劭弘 Vanilla\_Yukirin}
\institute{CJLU 2023 大数据}
\date{\today}

\begin{document}

\begin{frame}
  \titlepage
\end{frame}

\begin{frame}{目录}
  \tableofcontents
\end{frame}


\section{Matlab介绍}
\begin{frame}[fragile]{Matlab简介}
  Matlab(矩阵实验室)是由美国MathWorks公司开发的一种高级技术计算语言和交互式环境,广泛应用于科学计算、数据分析、算法开发和可视化等领域。
\end{frame}

\begin{frame}[fragile]{为什么选MATLAB?}
\begin{columns}[T]
\column{0.5\textwidth}
\textbf{MATLAB}
\begin{itemize}
  \item 功能丰富:\textbf{内置}数学计算函数和工具箱(各领域的函数库)、交互式工作区、工程领域标准
  \item 多工具箱:不需要会环境配置,安装好后开箱即用
  \item 文档齐全,内置帮助文档(按F12即可查看函数详细用法),相比去问AI会更加精确且全面
  \item 付费软件,昂贵的授权费,每个工具箱还需要单独付费\footnote{其实我们学校之前是有订阅过的,但是嘛,这里有个悲伤的故事……},很多学校没有资金订阅
\end{itemize}

\column{0.5\textwidth}
\textbf{Python}
\begin{itemize}
  \item 功能丰富:有numpy、sklearn、pandas、matplotlib等科学计算库、jupyter notebook交互式工作区,但是需要会环境配置和手动安装
  \item 多开源包:需要会环境管理/虚拟环境(anaconda、uv、venv等)
  \item 教程齐全:作为一门热门语言,网络上教程很多,问LLM也能得到不错的答案
  \item 免费软件:免费、开源、社区活跃,每个人都能自由的使用(freeware)
\end{itemize}
\end{columns}
\end{frame}

\begin{frame}{为什么数模初学者用MATLAB?}
  MATLAB 是工程领域的计算器+画图板+方程求解器——5分钟从数据到图表,无需折腾环境配置。安装完立即算矩阵、画图、解方程,让新生专注算法而非工具;而 Python 需要先学虚拟环境管理,对完全零基础的大一同学是额外负担。但自己装 Python 的编程环境常见的环境问题(依赖冲突、版本不兼容)往往劝退初学者。
\end{frame}

\begin{frame}{Matlab版本选择}
  Matlab相邻版本之间差别并不大。且近年更新的高级功能做数模一般用不上,所以只要不用过于老的版本(推荐使用 $\ge 2016b$),语法都是通用的。
  
  \textbf{ab尾缀的含义:}
  \begin{itemize}
    \item \textbf{a}:上半年发布的版本,通常在3月份发布
    \item \textbf{b}:下半年发布的版本,通常在9月份发布
  \end{itemize}
  
  通常来说,一个a版本发布后,经过几个月的使用和反馈,MathWorks会修复一些Bug,并在同年秋季的b版本中使其更加稳定。所以\textbf{推荐使用b尾缀的版本}。
  
  接下来的教程是基于Matlab R2024b的,如果你的版本不是2024b也没有关系,操作完全相同。
\end{frame}

\begin{frame}[fragile]{Matlab安装实操演示}
  自行下载matlab,网盘链接在钉钉群中
  
  如果下载速度过慢,可以带u盘下课找我拷贝
  
  \textit{TODO: 安装matlab过程截图}
\end{frame}

\section{Matlab的基础使用}

\subsection{认识Matlab基本界面}

\begin{frame}[fragile]{Matlab基本界面介绍}
  双击Matlab图标,等待一会儿,就会出现白色的Matlab软件窗口。

  工作区(当前工作目录)|命令行窗口|文件命名规范|运行代码
\end{frame}

\begin{frame}[fragile]{Matlab工作区}
  工作区,顾名思义就是进行工作的地方。

  建议先更换工作区到自己准备好的地方,做好文件管理,不要到处都散乱着代码文件。

  \textit{TODO: 详细过程介绍+截图}
\end{frame}




\section{简单数学模型的介绍}

























\section{问题背景}
\begin{frame}{研究问题}
  给定序列 $a_1,\dots,a_n$,求连续子段最大和(Maximum Subarray)。
\end{frame}

\section{算法}
\begin{frame}{Kadane 算法}
  递推:$dp[i]=\max(a_i, dp[i-1]+a_i)$,答案为 $\max_i dp[i]$。
\end{frame}

\begin{frame}[fragile]{示例代码(Python)}
\begin{lstlisting}[language=Python]
def max_subarray(a):
    best = cur = a[0]
    for x in a[1:]:
        cur = max(x, cur + x)
        best = max(best, cur)
    return best
\end{lstlisting}
\end{frame}

\begin{frame}[fragile]{示例代码(Matlab)}
\begin{lstlisting}[language=Matlab]
function ans = max_subarray(a)
    best = a(1);
    cur = a(1);
    for i = 2:length(a)
        cur = max(a(i), cur + a(i));
        best = max(best, cur);
    end
    ans = best;
end
\end{lstlisting}
\end{frame}

\begin{frame}{结论}
  Kadane 算法:时间 $O(n)$,空间 $O(1)$。
\end{frame}

% 新增:行内等宽(不需要 fragile)
\begin{frame}{行内等宽:\texttt{...}}
这里演示行内等宽:\texttt{for i = 1:n},适合短的、简单的代码或变量名。
\end{frame}

% 新增:listings 的行内(颜色来自 lstset)
\begin{frame}[fragile]{使用 \lstinline|...|}
演示 listings 行内:\lstinline|disp('hello')|,适合需要 listings 语法高亮的短代码片段。
\end{frame}

% 新增:单行短代码块(verbatim)
% \begin{frame}[fragile]{单行短代码块(verbatim)}
% 单行或几行的简短代码可以用 verbatim 环境:
% \begin{verbatim}
% x = 1; y = x^2; disp(y)
% \end{verbatim}
% \end{frame}

\end{document}